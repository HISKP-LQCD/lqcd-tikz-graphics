% Copyright © 2017-2018 Martin Ueding <dev@martin-ueding.de>
% Licensed under CC-BY 4.0

\documentclass{scrartcl}

\pagestyle{empty}

\usepackage{tikz}

\usetikzlibrary{arrows.meta}
\usetikzlibrary{calc}
\usetikzlibrary{decorations.markings}
\usetikzlibrary{decorations}
\usetikzlibrary{positioning}

\begin{document}

\tikzset{
    gauge link/.style={
        %thick,
    },
    link arrow/.style={
        decoration={
            markings,
            mark=at position 0.1 with {\arrow{Straight Barb}},
        },
        postaction={decorate},
    },
    link arrow midway/.style={
        decoration={
            markings,
            mark=at position 0.5 with {\arrow{-{Latex}}},
        },
        postaction={decorate},
    },
    gauge link arrow/.style={
        gauge link,
        link arrow midway,
    },
    lattice point/.style={
        draw,
        fill,
        circle,
        minimum size=1.7mm,
        inner sep=0,
    },
}

\begin{tikzpicture}

    \foreach \x in {0,1,2}
    {
        \foreach \y in {0,1,2}
        {
            \coordinate (n\x\y) at ($\x*(20mm, 0)+\y*(0, 20mm)$);
        }
    }

    \draw (-0.2, 0) -- (4.2, 0);
    \draw (-0.2, 2) -- (4.2, 2);
    \draw (-0.2, 4) -- (4.2, 4);

    \draw (0, -0.2) -- (0, 4.2);
    \draw (2, -0.2) -- (2, 4.2);
    \draw (4, -0.2) -- (4, 4.2);

    \newcommand\cloverlink{
        \draw[link arrow, thick] (3, 2.17) .. controls (3.83, 2.17) and (3.83, 2.17) .. (3.83, 3);
    }

    \newcommand\linkmargin{1.7mm}
    \newcommand\cloverplaquette{
        \cloverlink

        \begin{scope}[transform canvas={rotate around={90:(3, 3)}}]
            \cloverlink
        \end{scope}
        \begin{scope}[transform canvas={rotate around={180:(3, 3)}}]
            \cloverlink
        \end{scope}
        \begin{scope}[transform canvas={rotate around={270:(3, 3)}}]
            \cloverlink
        \end{scope}
    }

    \cloverplaquette
    \begin{scope}[transform canvas={rotate around={90:(n11)}}]
        \cloverplaquette
    \end{scope}
    \begin{scope}[transform canvas={rotate around={180:(n11)}}]
        \cloverplaquette
    \end{scope}
    \begin{scope}[transform canvas={rotate around={270:(n11)}}]
        \cloverplaquette
    \end{scope}

    \foreach \x in {0,1,2}
    {
        \foreach \y in {0,1,2}
        {
            \node[lattice point] at (n\x\y) {};
        }
    }

    \node[left=1.2mm of n11, fill=white, circle, inner sep=1pt, opacity=0.8] {\phantom{$n$}};
    \node[left=1.2mm of n11, circle, inner sep=1pt] {$n$};

    %\coordinate (axes-base) at ($(n01)+(-20mm, -5mm)$);
    %\draw[->] ($(axes-base)+(-2mm, 0)$) -- ++(10mm, 0) node[right] {$\mu$};
    %\draw[->] ($(axes-base)+(0, -2mm)$) -- ++(0, 10mm) node[above] {$\nu$};

    \node[above right=1mm of n22] {$n + \mu + \nu$};
    \node[above left=1mm of n02, white] {$n + \mu + \nu$};
    \node[above=1mm of n12] {$n + \nu$};
    \node[right=1mm of n21] {$n + \mu$};

    \node at ($0.5*(n11)+0.5*(n22)$) {$P_{\mu\nu}(\vec n)$};

    \draw[draw=white, fill=white] ($(n11)+(50mm, 0)$) circle (1mm);
    \draw[draw=white, fill=white] ($(n11)+(-50mm, 0)$) circle (1mm);

\end{tikzpicture}

\end{document}
