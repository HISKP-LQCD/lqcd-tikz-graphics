% Copyright © 2017 Christopher Helmes <helmes@hiskp.uni-bonn.de>
% Licensed under CC-BY 4.0

\documentclass{scrartcl}

\pagestyle{empty}

\usepackage{tikz}

\begin{document}

Broken

% \begin{tikzpicture}[node distance=1.5cm]
% 	% A picture showing the direct diagram of a 4pt correlation function. Need two vertices at tf and ti, and four quarklines, also the vertices should be named with the vertex factors and tf and ti as well 
% 	% Vertices: Two Kaon vertices and two pion vertices. represent with filled, empty circle
% 	% style for nodes
% 	% styles for the mesons
% 	\tikzstyle{strange}=[shape=circle, draw=black, fill=black!30]
% 	\tikzstyle{light}=[shape=circle, draw=black]
% 	% styles for the quarklines
% 	\tikzstyle{itof}=[thick,decoration={markings,mark=at position 0.5 with {\arrow{>}}},postaction=decorate]
% 	\tikzstyle{ftoi}=[thick,decoration={markings,mark=at position 0.5 with {\arrow{<}}},postaction=decorate]
% 
% 	\node[strange] at (0,0.5) (k_ti)[label=above:$x_1$,label=left:$\mathds{1}$] 		 			{};
% 	\node[light] (pi_ti) 			[below of=k_ti,label=below:$x_2$,label=left:$\mathds{1}$] {};
% 	\node[strange] at (2,0.5) (k_tf)[label=above:$y_2$,label=right:$\mathds{1}$] 		 			{}
% 	
% 	edge [ftoi, bend left=45] node[auto,swap]{$s$}(k_ti)
% 	edge [itof,bend right=45] node[auto,swap]{$u$}(k_ti);
% 	\node[light] (pi_tf) 			[below of=k_tf,label=below:$y_1$,label=right:$\mathds{1}$] {}
% 	edge [ftoi, bend left=45] node[auto,swap]{$u$}(pi_ti)
% 	edge [itof,bend right=45] node[auto,swap]{$u$}(pi_ti);
% 	
% 	% Quarkline 1
% 	%\draw (0,0 .. (2,0);
% \end{tikzpicture}
\end{document}
